\documentclass{beamer}
\setbeamercovered{transparent}

\input{preamble}
\usepackage{color}

\begin{document}
\title{3D  Parameterization: \\ PolyCube and CubeCover}
\author{Tengfei Jiang}

\newcommand{\FPP}[2]{\frac{\partial #1}{\partial #2}}
\begin{frame}
  \titlepage
\end{frame}

\section{Outline}
\begin{frame}{Outline}
  \begin{enumerate}
    \item Introduction
    \item CubeCover and PolyCube
    \item Conclusion
  \end{enumerate}
\end{frame}

\begin{frame}{What we want to do?}
  \begin{figure}
    \begin{picture}(1,0.1)(0,0)
      \PutPic{0.1,-0.2}{height=0.5}{cirque.png}
      \put(0.45,0.1){\large{$f:V\to\Omega$}}
    \end{picture}
  \end{figure}
\end{frame}

\begin{frame}{What we need to do?}
  Analogous to 2D parameterization, a good 3D parameterization needs a suitable frame-field.

  Which can be defined as follows:
  \begin{equation*}
  \begin{gathered}
    \min \int_{V}\|\nabla f - X\|^2 dV \\
    \mbox{s.t.}\quad C = 0
  \end{gathered}
\end{equation*}
Where $C=0$ is the boundary condition, and $X$ is the frame-field.
Then the most important thing is to generate a good $X$: 

\color{red}{\textit{smooth, topology well defined, controllable,...}}.
\end{frame}

\begin{frame}{Frame-Field Unknown}
  \begin{enumerate}
    \item Smooth : It has been done.
    \item Topology: \textit{CubeCover} has found a few things, but it's still left unknown on how to place singularities.
    \item Controllable: Depends on topology. Also unknown.
  \end{enumerate}
\end{frame}

\section{CubeCover} 
\begin{frame}{CubeCover}
  \begin{figure}
    \begin{picture}(1,0.1)(0,0)
      \PutPic{0.15,-0.05}{height=0.35}{meta-mesh-2.png}
      \put(0,-0.1){A good frame-field should statisfy the topology}
      \put(0,-0.15){constaint: singularities should be placed well.}
    \end{picture}
  \end{figure}
\end{frame}


\begin{frame}{Frame-Field Construction}
\begin{figure}
  \begin{picture}(1,0.1)(0,0)
    \PutPic{0.15,-0.05}{height=0.35}{frame-field-trilinear.png}
    \put(0,-0.1){By using the trilinear interpolation, it will get a frame on each point.}
    \put(0,-0.15){So that each tet has a frame at the barycenter.}
    \put(0,-0.2){\color{blue}{\textit{It looks like a simple smoothing process.}}}
  \end{picture}
\end{figure}
\end{frame}

\begin{frame}{Singularity}
  \begin{figure}
    \begin{picture}(1,0.1)(0,0)
      \PutPic{0.15,-0.15}{height=0.45}{singularity-0.png}
      \put(0.6,0.25){$x$}
      \put(0.57,0.23){$y$}
      \put(0.65,0.17){$x$}
      \put(0.62,0.14){$-y$}
      \put(0.51,0.16){$y$}
      \put(0.53,0.14){$-y$}
      \end{picture}
  \end{figure}
\end{frame}


\begin{frame}{Frame-Field Construction}
\begin{figure}
  \begin{picture}(1,0.1)(0,0)
    \PutPic{0,0.1}{height=0.2}{singularity.png}
    \put(0.3,0.25){The major character of frame field is }
    \put(0.3,0.2){to control the singularities.}
    \PutPic{0,-0.15}{height=0.2}{orientation.png}
    \put(0.3,0){\textit{matching matrix} $\prod_{st}$ defines the difference}
    \put(0.3,-0.05){ from orientation of s to orientation of t. }
  \end{picture}
\end{figure}
\end{frame}

\begin{frame}{Inner-Singularity}
  \begin{figure}
    \begin{picture}(1,0.1)(0,0)
      \PutPic{0,0}{height=0.25}{singularity-matching-matrix.png}
      \put(0.3,0.2){The singularity type of edge $e$ is defined as:}
      \put(0.3,0.15){$type(e,t_0):= \prod_{t_kt_0}\circ\prod_{t_{k-1}t_k}\circ...\circ\prod_{t_1t_2}\circ\prod_{t_0t_1}$}
      \put(0.3,0.1){The tets ($t_0...t_k$) is around the edge $e$.}
      \put(0,-0.05){\textbf{Theorem:}}
      \put(0.2,-0.05){Let $f$ be a parameterization where no tet is mapped}
      \put(0,-0.1){to degenerated tet(a tet with vanishing volume). }
      \put(0,-0.15){Then the \textit{type} of each edge is either the identity or a rotation of $90^{\circ}$}
      \put(0,-0.2){around one of the coordinate axes.}
    \end{picture}
  \end{figure}
\end{frame}



\begin{frame}{An observation which may be usefull}
\begin{figure}
  \begin{picture}(1,0.1)(0,0)
    \put(0,0.3){\textbf{Theorem:}}
    \put(0,0.25){Let $p$ be an inner vertex of a non-degenerated hexahedral mesh }
    \put(0,0.2){and $e_i$ its adjcaent edges. Then: \quad  $\sum_i(6-valence(e_i))=12$}
    \put(0,0.15){Where valence of $e$ is the number of adjacent hexahedra.}
    \PutPic{0.1,-0.2}{height=0.3}{observation-0.png}
    \PutPic{0.5,-0.2}{height=0.25}{observation-1.png}
  \end{picture}
\end{figure}
\end{frame}

\begin{frame}{An observation which may be usefull}
The above theorem indicates that:
\begin{enumerate}
  \item A singularity edge will not simply start or end  interior. They usually meet at \textit{node points} or hit the bounding surfaces.
  \item Not all situations which can fulfill above equation exist in real life. \color{red}{Not complete.}
\end{enumerate}
\begin{figure}
  \begin{picture}(1,0.1)(0,0)
    \PutPic{0,-0.1}{height=0.25}{valence-6-2.png} 
    \PutPic{0.5,-0.1}{height=0.25}{valence-6-3.png} 
    \end{picture}
\end{figure}
\end{frame}

\section{PolyCube}
\begin{frame}{PolyCube}
  \begin{figure}
    \begin{picture}(1,0.1)(0,0)
      \PutPic{0,-0.05}{height=0.3}{deformation.png}
    \end{picture}
  \end{figure}
\end{frame}

\begin{frame}{Deformation}
  \begin{figure}
    \begin{picture}(1,0.1)(0,0)
      \PutPic{-0.05,0}{height=0.2}{rotation.png}
      \PutPic{0.5,0}{height=0.2}{position.png}
    \end{picture}
  \end{figure}     

\TwoColumns{0.4}{
    \begin{block}{Rotation}
        Align the model surface normal with the global axes: $\pm X, \pm Y, \pm Z$
      \end{block}
  }{0.4}{
    \begin{block}{Position}
        Align each PolyCube face with an appropriate axis.
    \end{block}
  }
\end{frame}

\begin{frame}{Limitation}
  \begin{figure}
    \begin{picture}(1,0.1)(0,0)
      \PutPic{0.1,0.05}{height=0.15}{non-planar.png}
    \end{picture}
   \end{figure}

   Enforcing the planarity constraint will result in large distortion. That's because it's not a plane around the pink point.

   It needs a chart cutting and inserting. But the chart insertion process is not guaranteed to produce a valid PolyCube.

   \color{red}{No topology consistency guarantee}

\end{frame}



\section{Conclusion}
\begin{frame}{Conclusion}
The two method are both semi-automatic parameterization.
\begin{itemize}
\item PolyCube: 
  \begin{itemize}
    \item All-hex mesh
    \item Singularity-free interior
    \item No theoretical guarantee on producing a valid PolyCube
    \end{itemize}

\item CubeCover: 
  \begin{itemize}
    \item Singularity controllable but need to manually generate meta-mesh
    \item Lack of frame-field smoothing
    \item There is no sufficient condition on finding a frame-field which adheres to given singularities.
    \end{itemize}
\end{itemize}
%\[\int_{birth}^{death}study(t)dt = life\]
\end{frame}
\end{document}